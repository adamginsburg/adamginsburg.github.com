\documentclass{article}
\usepackage{url}
\usepackage{pbox}
%\usepackage[utf8]{inputenc}
\usepackage{listliketab}
\usepackage[margin=1.0in]{geometry}
\usepackage{natbib}  % Requires natbib.sty, available from http://ads.harvard.edu/pubs/bibtex/astronat/
%\usepackage[sorting=ydnt]{biblatex}
%\setlength{\bibsep}{0pt}
\citestyle{aa}  % (Author YYYY) references instead of (Author, YYYY)
\renewcommand{\refname}{Publications:}
\newcommand\apj{{ApJ}}%
\newcommand\apjl{{ApJ}}%
\newcommand\aj{{AJ}}%
\newcommand\aap{{A\&A}}%
\newcommand\apss{{Ap\&SS}}%
\newcommand\araa{{ARA\&A}}%
\newcommand\nat{{Nature}}%
\newcommand\aapr{{A\&A~Rev.}}%
\newcommand\mnras{{MNRAS}}%
\newcommand\apjs{{ApJS}}%
\newcommand\pasp{{PASP}}%
\usepackage{titlesec}
\titleformat{\section}{\large\bfseries}{\thesection}{}{}
\pagestyle{empty} % no page numbers 
\newcommand{\myname}[1]{\textbf{{\normalsize #1}}}



\begin{document}
\begin{center}
{\large Dr. Adam G. Ginsburg}\\
\par Fellow, European Southern Observatory
\par ESO Headquarters
\par Karl-Schwarzschild-Str. 2
\par 85748 Garching bei Muenchen, Germany
\par Phone: +49~0157-8719-2507
\par E-mail: adam.ginsburg@eso.org / adam.g.ginsburg@gmail.com
% why is the hspace hack necessary?  because CENTER, not CENTERING
\par Website: \url{www.adamgginsburg.com}
\end{center}
  
\section*{Educational Background: }
\begin{tabular} {cll}
    \textbullet & 2013~PhD Astrophysics & University of Colorado, Boulder \\
    \textbullet & 2009~M.S. Astrophysics & University of Colorado, Boulder \\
    \textbullet & 2006~B.S. Astrophysics & Rice University \\
\end{tabular}


\section*{Professional Employment:}
\begin{listliketab}
    %\storetstyleof{itemize}
    \begin{tabular}{clll}
    \textbullet & 2013 - present & Fellow  & European Southern Observatory\\
                &              &         & Garching, Germany \\
    \textbullet & 2007 - 2013 & Graduate Research Assistant & Center for Astrophysics and Space Astronomy, \\
                &           &                             & University of Colorado, Boulder, CO \\
    \textbullet & 2010 - 2013 & Instructor  & Department of Astrophysical and Planetary Sciences, \\
                                         &&& University of Colorado, Boulder, CO \\
    \textbullet & 2007 - 2011    & Teaching Assistant & Department of Astrophysical and Planetary Sciences, \\
                                                  &&& University of Colorado, Boulder, CO \\
    \textbullet & 2007 & Research Assistant & Department of Physics and Astronomy, \\
                                          &&& University of Denver, Denver, CO \\
    \end{tabular}
\end{listliketab}

\section*{Areas of Research: }
\begin{itemize}
    \item The astrophysics of massive star formation and the interstellar
        medium, with a focus on observations of proto-cluster
        clumps, massive outflows, and turbulence. 
    \item Observing with single-dish heterodyne and continuum millimeter
        instruments, single dish and synthesis array radio imaging,  and
        optical and near infrared imaging and spectroscopy. 
    \item The development of software tools for the analysis and visualization
        of diffuse and extended emission, spectral data cubes, and large
        astromical data sets.
\end{itemize}

%Ginsburg, Adam, Bally, John, Williams, Jonathan 2011, MNRAS, in press 
%   “JCMT HARP CO 3-2 Observations of Molecular Outflows in W5” 
%Ginsburg, Adam; Darling, Jeremy; Battersby, Cara; Zeiger, Ben; Bally, John 2011, ApJ, 736, 149 
%   “Galactic H2CO Densitometry I: Pilot survey of Ultracompact HII regions and methodology” 
%Schlingman, Wayne M.; Shirley, Yancy L.; Schenk, David E.; Rosolowsky, Erik; Bally, John; Battersby, Cara; Dunham, Miranda K.; Ellsworth-Bowers, Timothy P.; Evans, Neal J., II; Ginsburg, Adam; Stringfellow, Guy 2011, ApJS, 195, 14
%   “The Bolocam Galactic Plane Survey V: HCO+ and N2H+ Spectroscopy of 1.1 mm Dust Continuum Sources”
%van Aarle, E.; van Winckel, H.; Lloyd Evans, T.; Ueta, T.; Wood, P. R.; Ginsburg, A. G. 2011, A&A, 530, 90
%   “The optically bright post-AGB population of the LMC”
%Battersby, C.; Bally, J.; Ginsburg, A.; Bernard, J. -P.; Brunt, C.; Fuller, G. A.; Martin, P.; Molinari, S.; Mottram, J.; Peretto, N.; Testi, L.; Thompson, M. A. 2011, accepted
%   “Characterizing Precursors to Stellar Clusters with Herschel”
%Bally, John; Aguirre, James; Battersby, Cara; Bradley, Eric Todd; Cyganowski, Claudia; Dowell, Darren; Drosback, Meredith; Dunham, Miranda K.; Evans, Neal J., II; Ginsburg, Adam; Glenn, Jason; Harvey, Paul; Mills, Elisabeth; Merello, Manuel; Rosolowsky, Erik; Schlingman, Wayne; Shirley, Yancy L.; Stringfellow, Guy S.; Walawender, Josh; Williams, Jonathan 2010, ApJ, 721, 137
%    “The Bolocam Galactic Plane Survey IV: 1.1 and 0.35 mm Dust Continuum Emission in the Galactic Center Region”
%Yan, Chi-Hung; Minh, Y. C.; Wang, Shiang-Yu; Su, Yu-Nang; Ginsburg, Adam, 2010, ApJ, 720,1-8 
%    “Star-forming Region Sh 2-233IR. I. Deep Near-infrared Observations toward the Embedded Stellar Clusters” 
%Aguirre, J, Ginsburg, A.G., Dunham, M.K., Drosback, M. and the BGPS collaboration 2010 ApJ accepted 
%    “The Bolocam Galactic Plane Survey -- I. Survey Description and Data Reduction” 
%Dunham, M.K., Rosolowsky, E., Evans, N. J, Cyganowsky, C. J., and the BGPS collaboration 2010, ApJ, 717, 1157
%    “The Bolocam Galactic Plane Survey -- III. Characterizing Physical Properties of Massive Star-Forming Regions in the Gemini OB1 Molecular Cloud” 
%Ginsburg, A. G.; Bally, J.; Yan, C.; Williams, J. P. 2009, ApJ, 707, 310-327 
%    “Outflows and Massive Stars in the protocluster IRAS 05358+3543” 
%Rosolowsky, E., Dunham, M. K., Ginsburg, A.G., Bradley, E. T., and the BGPS collaboration 2009, ApJS, 188, 123-138 
%    “The Bolocam Galactic Plane Survey -- II. Catalog of the Image Data” 
%Rubin, D. and the SAGE collaboration, 2009, A&A, 494, 647-661 
%    “A spatially resolved study of photoelectric heating and [C II] cooling in the LMC. Comparison with dust emission as seen by SAGE” 
%Van de Steene, G.C., Ueta, T., van Hoof, P.A.M., Reyniers, M., Ginsburg, A.G. 2008, A&A, 480, 3, 775-783 
%    “Kinematics and H2 morphology of the multipolar post-AGB star IRAS 16594-4656”



\setlength{\extrarowheight}{4pt}
\section*{Honors/Awards: }
\vspace{-12pt}
\begin{tabular}{cll}
    \textbullet & 2013 & European Southern Observatory Postdoctoral Fellowship \\
    \textbullet & 2011 & University of Colorado Chance Irick Cooke Fellowship for Excellence in Research \\
    \textbullet & 2010 & NRAO Green Bank Student Observing Support  \\
    \textbullet & 2010 & NSF GRFP Honorable Mention
 \\
    \textbullet & 2009 & NSF GRFP Honorable Mention
 \\
    \textbullet & 2008 & NSF GRFP Honorable Mention
 \\
    \textbullet & 2008 & NRAO Photo Contest First Prize \\
    \textbullet & 2008 & University of Colorado Astrophysical and Planetary Sciences Excellence in Teaching award  \\
    \textbullet & 2006 & National Radio Astronomy Observatory - summer REU with David Meiers  \\
\end{tabular}

\setlength{\extrarowheight}{4pt}
\section*{Academic Advising: }
\vspace{-12pt}
\begin{tabular}{p{0.75in}p{1.2in}lp{2.6in}}
    Date & Program & Student & Project \\
    \hline
    Fall 2013 -\newline present & Ludwig-Maximilian University / ESO PhD Thesis Student & Anna Faye McLeod & FUSION: Comparison of hydrodynamic simulations and observations in nearby high mass star forming regions  \\
    Summer 2014 & Google Summer of Code & Simon Liedtke& New tools for \texttt{astroquery}: XMatch, SkyView, Atomic Line List \\
    Summer 2013 & Google Summer of Code & Madhura Parikh& A coherent API for \texttt{astroquery}, a python web database query toolkit \\
\end{tabular}

\setlength{\extrarowheight}{4pt}
\section*{Teaching: }
\vspace{-12pt}
\begin{tabular}{cll}
                & Date         & Course \\
                \hline
    \textbullet & Spring 2013  &   Instructor of ASTR 2600: Introduction to Programming for Astronomers (in IDL) \\
    \textbullet & Fall 2012    &   Instructor of ASTR 2600: Introduction to Programming for Astronomers (in IDL) \\
    \textbullet & Summer 2010  &   Co-Instructor of ASTR 1020: Stars and Galaxies \\
    \textbullet & Fall 2011    &   Co-Instructor of ASTR 6000: Graduate Seminar on the Interstellar Medium \\
    \textbullet & Fall 2011    &   Teaching Assistant for ASTR 3510: Astronomical Observing (imaging) \\
    \textbullet & Spring 2010  &   Teaching Assistant for ASTR 3520: Astronomical Observing (spectroscopy) \\
    \textbullet & Fall 2009    &   Teaching Assistant for ASTR 3510: Astronomical Observing (imaging) \\
    \textbullet & Fall 2008    &   Teaching Assistant for ASTR 3520: Astronomical Observing (spectroscopy) \\
    \textbullet & Spring 2008  &   Teaching Assistant for ASTR 3510: Astronomical Observing (imaging) \\
    \textbullet & Fall 2007    &   Teaching Assistant for ASTR 3520: Astronomical Observing (spectroscopy) \\
\end{tabular}

\setlength{\extrarowheight}{4pt}
\section*{Conferences and Workshops: }
\vspace{-12pt}
\begin{tabular}{cp{1.8in}p{1.5cm}p{3.0in}}
    Date & Meeting Name & Role & Talk or Poster Title \\
                \hline
    2015 &      Florence Simulation-Observation Workshop & Organized & CAMELOT: Comparing simulations and observations \\
    2015 &      ESO Central Molecular Zone workshop (2 days)   & Organized & Dense Gas Structure and Thermal Balance in the CMZ \\
    2015 &      Astropy Lorentz Center Workshop (5 days) & Talks \& unconferences & radio-astro-tools, astroquery, and spectral-cube \\ 
    2015 &      University of Munich Filaments Workshop (3 days) & Talk & W51: The most active star-forming complex in the Galaxy \\
    2015 &      Soul of High Mass Star Formation, Chile & Talk & The Density Structure of the W51 GMC \\
    2014 &      Workshop on the APEX CMZ 1 mm survey at MPIfR Bonn (1 day)   & Organized &  \\
    2014 &      ALMA Arc Node Retreat  & Talk & ALMA's first look at the extended Sgr B2 Cloud \\
    2014 &      Sexten Workshop: The Formation of Globular Clusters  & Talk & The Galactic population of young massive clusters \\
    2014 &      Sexten Workshop: The assembly of massive clusters  & Talk & The density of W51 and its protoclusters \\
    2014 &      Early Phase of Star Formation (EPoS 6)  & Talk & The density structure of The Brick \\
    2014 &      Early Phase of Star Formation (EPoS 6)  & Poster & The density structure of the W51 Giant Molecular Cloud \\
    2013 &      ISM Physical Processes in Garching  & Poster & A measurement of the turbulence driving parameter \\
    2013 &      .Astronomy 5  & Talk & Astroquery: A toolkit for remote data access in python \\
%    2013 &     IAU 303: The Galactic Center  & Poster&  \\
    2013 &      AAS 221  & Thesis Talk & Surveying massive star formation in the Galactic Plane \\
    2012 &      Galactic Scale Star Formation  & Poster& There are no starless massive proto-clusters in the first quadrant \\
    2012 &      Labyrinth of Star Formation  &  Talk& Surveying Pre-Stellar Gas with the BGPS (with an emphasis on what we don't see) \\
    2011 &      Milky Way  & Talk& The Bolocam Galactic Plane Survey \\
    2010 &      Stars to Galaxies  & Poster& Star Formation in Perseus Arm Complexes \\
    2010 &      AAS 217  &  Poster& Formaldehyde Densitometry of Dust Clumps: The shapes and densities of massive star forming regions \\
    2009 &      AAS 215  &  Poster& The Bolocam Galactic Plane Survey: Data, Early Results, and Future Directions \\
%    \textbullet & IRAM single-dish summer school  &2009&   \\
%    \textbullet & VLA synthesis imaging summer school and summer REU &2006&  \\
\end{tabular}

% \section*{Selected Talks:}
% \begin{tabular}{cp{1.8in}cp{3.5in}}
%     \textbullet & ESO Lunch Talk & 2013 & Examining Massive Cluster Formation with H2CO in W51 \\
%     \textbullet & MPIfR Lunch Talk & 2013 & Surveying Star Formation in the Galactic Plane  \\
%     \textbullet & CfA Lunch Talk & 2013 & Surveying Star Formation in the Galactic Plane  \\
% \end{tabular}


\setlength{\extrarowheight}{7pt}

\section*{Successful Observing Proposals to open TACs as PI:}
\begin{tabular}{p{0.75in}p{3.25in}p{0.65in}p{0.70in}}
                Telescope & Title & Time & Status \\
    \hline
    {\textbf{VLA    }\newline {\small 2014} } & VLA15A-164: Studying turbulence through the atomic-to-molecular transition & 3.3 hours & Approved \\
    {\textbf{GBT    }\newline {\small 2014} } & GBT14A-329: MUSTANG Galactic Plane survey: HCHIIs in the brightest massive proto-clusters & 14 hours & Approved \\
    {\textbf{ALMA   }\newline {\small 2014} } & Cycle 2: 2013.1.00269.S: Sgr B2 - The Proving Ground for Star Formation Theories & 6 hours & In Progress\\
    {\textbf{LOFAR  }\newline {\small 2014} } & Cycle 2: LC2\_006: A search for p-H2CO, a potential EoR contaminant, toward the Galactic Center, W43, W44, W49, and M82. & 8 hours & Observed (2014) \\
    {\textbf{APEX   }\newline {\small 2014} } & H2CO Thermometry of the CMZ to understand its low star formation rate & 250 hours & Observed (2014) \\
    {\textbf{GBT    }\newline {\small 2014} } & GBT14A-110/GBT12B-221: Density Measurements in G0.253+0.016: Pilot program for CMZ H2CO densitometry & 18 hours & Observed (2014) \\
    {\textbf{KPNO   }\newline {\small 2013} } & 2013A-0399: Star formation in the Central Molecular Zone: Massive Outflows in Sgr C & 6 hours & Observed (2013) \\
    {\textbf{EVLA   }\newline {\small 2013} } & 13A/064: Massive stars and ionized gas in the W51 complex & 13 hours,\newline 4 configs & Observed (2014) \\
    {\textbf{Arecibo}\newline {\small 2012} } & A2854: Density Map of the W51 Giant Molecular Cloud complex & 13 hours & In~Press:~A\&A~2014 \\
    {\textbf{GBT    }\newline {\small 2010} } & GBT10B-019: Densitometry of young star-forming complexes throughout the Galaxy & 120 hours & Published: 2013ApJ...779...50G \\
    {\textbf{Arecibo}\newline {\small 2010} } & A2584: Densitometry of young star-forming complexes throughout the Galaxy & 60 hours & Published: 2013ApJ...779...50G \\
    {\textbf{GBT    }\newline {\small 2009} } & GBT09C-049:	Measuring the dense gas mass fraction with H2CO absorption & 4 hours & Published: 2011ApJ...736..149G \\
\end{tabular}

\section*{Other relevant observing:}
I was an active user of the Apache Point Observatory (APO) while at the
University of Colorado, using a few dozen nights of TripleSpec, NICFPS, and DIS
time to examine outflows in Sh 2-233 (IRAS 05358+3543), W5, Orion BN/KL, W51,
and Sgr C.

I am Co-I on successful proposals to the VLT, Gemini, SOFIA, HST, ALMA, IRAM
30m, HHT/SMT, APEX, Herschel, GBT, VLA, and Arecibo.  Please ask me if you
would like the complete detailed list.

\section*{Computing:}
I am an active developer of a large variety of astronomical python software
tools and a contributor to \texttt{astropy} and its affiliates.  My github
profile (\url{github.com/keflavich}) contains a complete list of projects.
Below is a selection of my most popular packages:

\begin{itemize}
    \item \texttt{astroquery} (\url{astroquery.readthedocs.org}):
        a toolkit for querying internet-hosted astronomical databases.
    \item \texttt{pyspeckit} (\url{pyspeckit.bitbucket.org}): a software suite
        for visualizing and analyzing spectral line and spectral cube
        data
    \item \texttt{spectral-cube} (\url{spectral-cube.rtfd.org}): a library for the manipulation
        of radio spectral cube data
    %\item \texttt{radio-astro-tools} (\url{radio-astro-tools.github.io}): an
    %    umbrella organization that includes tools for the analysis of radio
    %    data, including ALMA and EVLA spectral cubes.
    \item \texttt{pyradex} (\url{github.com/keflavich/pyradex}):
        an object-oriented frontend to the popular RADEX radiative transfer code and
        its peers.
    \item \texttt{image-registration} (\url{github.com/keflavich/image_registration}):
        a package designed to determine and correct the offsets between images containing only
        diffuse emission
    \item \texttt{FITS-tools} (\url{github.com/keflavich/FITS_tools}):
        a collection of tools for slicing and reprojecting FITS images and cubes
\end{itemize}

\section*{Service:}
\begin{itemize}
    \item Organizer of the ``Python Coffee and Tutorial'' series at ESO, 2014
    \item Referee for the following journals:
        \begin{itemize}
            \item \textit{Science}
            \item \textit{Astrophysical Journal}
            \item \textit{Astronomy \& Astrophysics} (main journal and letters)
            \item \textit{Monthly Notices of the Royal Astronomical Society}
            \item \textit{Revista Mexicana de Astronom{\'i}a y Astrof{\'i}sica}
    \end{itemize}
    \item Panel chair for a recent NASA grant review panel
    \item ESO ALMA Fellow Duties.  Primary duties include
        software development and maintenance of the Quality Assurance Packager
        software
    \item Co-organizer of the December 2014 ALMA Postdoc Symposium in Tokyo
    %\item Member of the \texttt{astropy} (\url{astropy.org}) collaboration
    \item Member of the \texttt{montage} (\url{montage.ipac.caltech.edu}) Image
        Mosaic Engine users group
    \item Member of the CCAT ISM working group
    \item Member of the Next-Generation VLA (NGVLA) high mass star formation
        working group
\end{itemize}


%\section*{Outreach:}
%    Judge for Denver Metro Science Fair, 2/27/2008
%    Volunteer for middle school science day at the University of Denver, May 2007
%    Volunteer for University of Colorado Astronomy Day, 2008-2011
%    Volunteer for Sommers-Bausch Observatory Open Houses, 2007-
%    Vail Nature Center presentation 7/19/2008
%    Judge at Boulder Country Day science fair 12/10/2008
%    Judge at Centennial Middle School science fair 1/30/2009
%    <a href="outreach.htm"> Public presentations at REI September 2008, 2009</a>
%    Presentation to Boulder Astronomical Society 1/16/2010
%    Winter Park Star Safari, July 2011



\newpage
\nocite{*}
%\section*{Publications: }
\begin{footnotesize}
\bibliographystyle{apj_revchron}
%\bibliographystyle{nature}
\bibliography{cv}
\end{footnotesize}


\end{document}
