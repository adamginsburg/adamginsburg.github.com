% original TeX version by sombody
% 2002 Dec 12: LaTeX conversion by Amy Mioduszewski
% 2002 Dec 12: Minor format changes by Amy Mioduszewski and Greg Taylor
% 2002 Dec 13: Fixes by Amy in response to Richard Porcas comments
% 2002 Dec 18: More minor fixes by Amy in response to Patrick Charlot
% 2002 Dec 31: More minor changes by Amy re: Mike Garret plus improved
%              underlining
% 2003 Jan 18: More minor improvments to the underlining by Amy
% 2003 Jul 21-22: Added student info, rapid response science by Jim U.
% 2003 Aug 28: Some cleanup by Greg in response to Patrick Charlot
%
% This is the VLBI proposal cover sheet.  It is highly recommended 
% that you make a copy of this before editing.
% Lines beginning with ``%'' give instructions as to what to fill in
% or replace on immediately following lines.
%
% COVER SHEET, FRONT SIDE
%

%\documentclass[10pt]{article}
%\documentclass[preprint]{aastex}
\documentclass[11pt, preprint]{aastex}
\usepackage{amsmath}
\usepackage{graphicx}
\usepackage[left=1.75cm,top=3cm,right=2cm,bottom=.25cm,nohead,nofoot]{geometry}
\pagestyle{plain}
\def\unter#1{\hbox to #1{\hrulefill}}
\def\arcmin{$^\prime$}
\def\arcsec{$^{\prime\prime}$}
\newcommand{\formaldehyde}{H\ensuremath{_2}CO}
\newcommand{\hh}{H\ensuremath{_2}}
\newcommand{\oneone}{\ensuremath{1_{10}-1_{11}}}
\newcommand{\twotwo}{\ensuremath{2_{11}-2_{12}}}
\newcommand{\percc}{\ensuremath{\textrm{cm}^{-3}}}
\newcommand{\percms}{\ensuremath{\textrm{cm}^{-3}}}

\def\Figure#1#2#3#4{
\begin{figure}[htb]
  \epsscale{#4}
  \plotone{#1}
  \caption{#2}
  \label{#3}
\end{figure}
}
\def\FigureTwo#1#2#3#4#5{
\begin{figure}[htb]
\epsscale{#5}
\plottwo{#1}{#2}
\caption{#3}
\label{#4}
\end{figure}
}


\usepackage{natbib}  % Requires natbib.sty, available from http://ads.harvard.edu/pubs/bibtex/astronat/
\citestyle{aa}
\bibliographystyle{/Users/adam/papers/latexfiles/apj_w_etal}


\begin{document}

\normalsize

The distribution of dense molecular material on galactic scales is a necessary
component in understanding the evolution of molecular clouds and galactic star
formation rates.  While it is possible to explore the question of how
efficiently molecular clouds collapse into prestellar cores on galactic scales
in other galaxies and for small samples within the Milky Way, it has not been
possible to explore the details of cloud collapse in a nearby, unbiased sample
until recently.  We will use total mass estimates from the dust continuum in
combination with an estimate of the mass in high-density regions from
formaldehyde (\formaldehyde) 2cm absorption measurements to acquire an estimate
of the dense mass fraction in an unbiased sample of millimeter continuum
sources at different galactocentric radii.  

\formaldehyde\ can be used to measure the density in high-density regions.
Because of a collisional selection effect, above its critical density
\formaldehyde\ preferentially overpopulates the lower states of the \oneone\
through $5_{14}-5_{15}$ transitions.  These lines are cooled to excitation
temperatures lower than the CMB and can therefore be seen in absorption against
the CMB.  The \oneone/\twotwo\ absorption line ratio is very sensitive to
density of \hh\ at densities larger than $\sim10^{3.5}\ \percc$, allowing
measurements of the density to within $\sim 0.1$ dex with little sensitivity to
gas kinetic temperature; when density is `measured' with criticial density
based tracers the estimate can be off by as much as 2 orders of magnitude.  The
absorption line depth will be approximately proportional to the column of
\formaldehyde\ at the measured density, and therefore a simple geometric
scaling argument can be used to determine a size scale, $r = X_{\formaldehyde\
}^{-1} N(\formaldehyde) / n(\hh)$, where $X_{\formaldehyde}$ is the ratio of
\formaldehyde/\hh, which is typically $\sim10^{-9}$.  The density this ratio
will measure is heavily weighted towards high-density regions; the optical
depth curves shown in figure \ref{fig:densratio} are the approximate weighting
functions.

We can acquire mass estimates with the Bolocam Galactic Plane Survey
\citep[BGPS,][]{aguirre2009}, which is a 1.1mm survey of the Galactic plane
($-.5<b<.5$ over most of the plane) conducted at the Caltech Submillimeter
Observatory.  It includes a catalog of $\sim$8000 sources and flux
measurements.  The majority of the objects detected in the BGPS are thought to
be ``clumps'', intermediate between high-density prestellar cores and moderate
density Giant Molecular Clouds, with mean densities ranging from $10^2$ to
$10^4$ \percc.  A subsample of these sources have been followed up with NH$_3$
1-1 and 2-2 inversion transition observations, giving an accurate measure of
the gas temperature in the dense gas.
% Assuming dust kinetic
% temperatures of 20K, the survey is complete to objects of mean density
% $N_{H_2}\gtrsim10^3\percms$ out to 3 kpc, and $N_{H_2}>\sim10^{2.3}\percms$ out
% to 12 kpc.

%We use 1.1mm continuum measurements from the Bolocam Galactic Plane Survey
%\citep[BGPS,][]{aguirre2009}, which covered the first quadrant of the Galactic
%plane in the range $-0.5 < b < 0.5$, to acquire total gas mass estimates for
%our list of target sources. 

Millimeter continuum emission is an excellent optically-thin tracer of dust
emission.  Being optically thin, dust measurements at millimeter frequencies
are sensitive to the entire mass of molecular clouds, unlike molecular line
tracers like CO which become optically thick at relatively low column
densities.  Mass estimates from dust are therefore robust mass estimates of
molecular clumps and cores, but continuum tells little about the physical
properties of the star forming regions.  

With a size scale and density calculated from the \formaldehyde\ lines we can
compute the mass at the measured density, which should be a proxy for the
fraction of gas in prestellar cores.  We already have an estimate of the total
gas mass from the 1.1mm continuum.  The fraction of mass in dense regions will
constrain the time scale and efficiency with which molecular clouds collapse into
substructure that will form stars.  

For BGPS sources with a large fraction of dense gas, high-resolution followup
with interferometer arrays will be a good way to characterize this fraction
more precisely, but a survey of a large number of BGPS sources with
interferometers is technically infeasible.  A single BGPS source would take
$\sim$ 1 night to acquire good sensitivity to all of the dense gas in close-by
star forming regions, whereas a similar measurement with the GBT can be done in
5 minutes.  

% The \formaldehyde\ 2(12)-2(11) / 1(11)-1(10) line ratio is a sensitive estimator of the
% molecular gas (H$_2$) density and is robust to variations in abundance.
% \citet{mangum2008,mangum1993} have performed detailed studies of \formaldehyde\ as a
% densitometer, and have developed a grid of models covering a wide range of
% temperatures and densities that can be used to translate an optical depth
% measurement in the two \formaldehyde\ transitions to a number density of H$_2$ (figure \ref{fig:h2codens}).
% Uncertainties in the density measurement with no knowledge of the background
% continuum or kinetic temperature are only a factor of $\sim$2.  With knowledge
% of the background continuum and the local kinetic temperature, the density
% measurements become limited by measurement uncertainty at $\sim20\%$, assuming
% the large velocity gradient model used remains valid.  The method was confirmed
% using a sample of star forming galaxies in \citet{mangum2008}, but we will check 
% the validity of the method within the galaxy by looking at a few well-studied and
% understood sources (see sample section below).

% With a significant sample of sources at known distances with known densities, 
% we will be able to measure the gas mass and dust mass independently and
% therefore be able to measure the dust-to-gas ratio.  We will still have to
% assume a kinetic temperature, which can affect the dust mass measurement, but
% this can be resolved by other observations (e.g. NH$_3$ inversion transitions).
% We will also have to assume a source size, but since this size will be the same
% for the dust and gas measurements, it will not introduce any additional bias.

% Our long term plan is to extend this survey with Arecibo and GBT beam-matched
% observations of all BGPS sources within the range of Arecibo.  This complete
% sample of accurately measured masses and densities will serve as a calibration
% sample for the rest of the BGPS catalog by providing a stronger constraint on
% the dust/gas mass ratio.  It will also present the best-measured mass catalog
% from which a core/clump mass function can be determined.  Our survey will present
% an essential calibration for all millimeter Galactic plane surveys, including
% the JCMT SCUBA II, Herschel Hi-Gal, and APEX Atlasgal surveys, which combined
% will amount to thousands of hours of observing time on world-class telescopes.  

Furthermore, obtaining a comprehensive data set for \formaldehyde in dense clumps
across the galactic plane with allow calibration of \formaldehyde as a densitometer on
galactic scales.  \citet{mangum2008} have used \formaldehyde as a densitometer in a
large survey of nearby galaxies, detecting both transitions in five galaxies.
Using the ratio of the \oneone\ and \twotwo\ lines, they obtain a galaxy-wide
``density'' measurement of order $n(\hh)\approx10^5\percc$.  While these are
starburst galaxies with high average densities, the meaning of a ``galactic
average'' density determined from molecular absorption lines (which come from
clumpy media) is uncertain at best.  By using H$_2$CO to determine densities
across a broad sample of sources found with the BGPS survey, we can calibrate
Mangum et al.'s observations to obtain a more rigorous meaning for their
galactic average densities.  The star forming environment of BGPS clumps will
not be the same as the galaxies in the Mangum survey, but some clumps will
probably show similar environments, e.g., those closer to the Galactic Center.
By obtaining observations in different Galactic environments, we can determine
how metallicity and the local interstellar radiation field affect star formation...

% \Figure{darling2008_h2co_density}{\formaldehyde\ density measurements of star
% forming galaxies from LVG models \citep{mangum2008}.  The thick vertical bars represent the ratio
% between the \oneone\ and \twotwo\ transitions.  Dashed curved
% lines represent the intensities of the 6cm line.  The absorbed background is
% assumed to be the CMB.  The kinetic temperature is 40K.}{fig:h2codens}{0.4}
% \Figure{rosolowsky2009_densityvsdist}{Estimated BGPS source densities from the
% dust continuum as a function of distance.  Velocities for this sample were
% measured with the NH$_3$(1,1) and (2,2) inversion transitions in the $l=31$
% field.  Each source is represented twice. Plus signs represent the near
% distance, diamonds represent the far distance.  }{fig:densdist}{0.4}
\Figure{nh_vs_lineratio_10and40K}{The 6cm/2cm line ratio (solid) from large
velocity gradient (LVG) modeling as a function of density assuming T=10K (blue)
and 40K (green) and abundance of $10^{-9} \hh$.  The ratio provides only a weak
constraint on density for low densities without additional information, but for
these densities the optical depth (dashed, 10K blue, 40K green) of the \twotwo\
line is extremely low and it will not be seen.  Optical depth is plotted
assuming a filling factor of 1.}{fig:densratio}{0.5}

{\bf THE SAMPLE:}
In our pilot sample, there was only 1 clear non-detection out of 25 observed
sources, and the nondetection provided a strong limit on the source density
because it was detected by Arecibo...

Our sample will be selected from BGPS sources in the range l=32 to 75 and l=170
to l=212 in order to cover the range coincident with that availabe to the
Arecibo radio telescope.  We will select a sample of 300 sources from bright,
active complexes containing HII regions, quiescent infrared dark clouds,
filamentary structures, and other object classes in between.  The pilot survey
was limited to UCHII regions, which represent a late stage in the star formation
process.

%\citet{araya2002} performed a survey of 21 Galactic HII regions in the
%\formaldehyde\ 6cm ($\nu_0 = 4.8296594$ GHz) absorption line with the Arecibo
%telescope.  \formaldehyde\ absorption was detected in all of these sources with
%high S/N in 5 minutes per pointing.  Only a single \formaldehyde\ absorption
%line was detected in each source, whereas along most lines of sight in the
%inner galactic plane $^{13}$CO is seen in emission at multiple velocities
%\citep{jackson2006}.  This is probably an indication that \formaldehyde\ is
%only associated with the dense gas that we also expect to dominate the mass
%measurement in a BGPS source.
%
%\citet{araya2004} surveyed an additional 15 sources with known weak 6cm continuum
%emission and found absorption against the CMB in all 15 with the same 5 minute 
%exposure time.  We include this sample as a more representative sample of the
%BGPS because we do not expect all BGPS sources to contain HII regions. 

We select this sample to take advantage of simultaneous Arecibo observations so
that we may do beam-matched observations of the two \formaldehyde\ transitions
to be certain we are sampling the same population of sources.  The GBT and
Arecibo beams are well-matched at these two transitions.  The Arecibo beam at
4.8 GHz and the GBT beam at 14.48 GHz are both $ \sim$52\arcsec .  Beam
matching guarantees that we are sampling the same sightline and is therefore
essential to the success of the survey in the inner Galactic plane, where the
source density is very high.  The GBT is the essential instrument for this
project both because of the beam size and the sensitivity - a large sample
would be impractical with other large telescopes that have much lower
efficiencies in Ku band.  

%This initial sample will serve as an initial science sample and a proof of
%concept; a future study will take full advantage of the BGPS to make an
%unbiased selection of millimeter sources within Arecibo's observing range.

%Herschel's beam at 520 microns is 30.6",
%and the HiGal guaranteed time mission will be performing a Galactic plane
%survey similar to the BGPS in the first few years of the mission.  

% ... [this paragraph is probably unnecessary but it's somewhat interesting] ...
% \citet{dickel1996} made precise density measurements of arcsecond-scale
% features within W3(OH).  However, at arcsecond scales there are many reasonable
% methods to measure densities, and variations in \formaldehyde\ column due to freezeout
% are possible confusing factors.  \citet{dickel1986} performed calculations of
% density and abundance as a function of the 2cm/6cm optical depth ratio, but so
% far their work has only been applied on small scales to VLA observations.  The
% VLA will be without a U-band receiver until the end of EVLA commissioning, so
% additional high spatial resolution observations of the 2cm transition will have
% to wait.
% ...

% Some of our 26 sources, e.g. G34.26+.15, are very well-studied in multiple
% molecular transitions and millimeter continuum.  These sources will provide a
% test sample which we can use to confirm the validity of the density
% and dense mass fraction measurements.  

% In a future study, we will do beam-matched observations with GBT and Arecibo of
% the \citet{reid2009} sample of cores with very accurate distances to remove the
% distance as a possible ambiguity.  We have chosen the Araya sample in this study
% because one transition has already been observed in these sources.

% Once the technique is proved, we will use it on all sources available to both
% Arecibo and GBT in the BGPS sample ($\sim$ 500-1000 sources?).  This accurate
% sample will provide a direct measurement of the cloud/clump/core mass function.


{\bf TECHNICAL JUSTIFICATION}

{\it RFI:} The range 14.285-14.835 GHz  is regulated to avoid interference because of
the presence of the formaldehyde line we will be observing.

{\it Instrument:} We will use the GBT Ku-band spectrometer with 4 spectral windows
centered at 14.488479 (\formaldehyde), 14.1315 (H77$\alpha$), 14.693
(H76$\alpha$), and 15.2846 (H75$\alpha$) GHz using both polarizations (no
cross-products) with a bandwidth of 12.5 MHz, a channel width of 3.058 kHz, and
9-level sampling.  We will operate in nodding mode.

{\it Resonances:}  None of our target spectral lines fall near the two Ku band 
resonances at 12.875 $\pm0.0081$ and 12.885 $\pm0.0071$ GHz.

{\it Time:} Our sensitivity depends both on the strength of the continuum source and
the strength of the absorption feature.  We assume the continuum source is the
CMB, which provides a lower limit to the continuum source strength that will be
exceeded when an HII region provides the continuum.

In a 5 minute exposure (5 on - no off is required with the two-beam Ku
receiver) with the GBT spectrometer at 14.488479 GHz using two polarizations
(no cross-products) with a channel with of 3.058 kHz (.064 km/s) assuming a
system temperature of 30K (75th percentile) we will achieve an RMS sensitivity
of 20.80 mK (10.89 mJy), which translates to an optical depth of the
\formaldehyde\ \twotwo\ line of 0.008.  The high spectral resolution is not
necessary for the stated goal of measuring the density but will be useful for
measuring the line profile, which may prove useful in comparison with other
line profiles for our high S/N observations.  The data can be resampled in
post-processing to 1 km/s  resolution to yield an RMS sensitivity of 5.20 mK or
$\tau=0.002$.  

%The faintest object detected in the \citet{araya2004} survey had optical depth
%(assuming absorption against the CMB) $\sim 0.03$ ($T\sim-82$mK), but the
%majority were detected at higher significance.  In the HII sample
%\citep{araya2004}, the smallest signal was five times higher at $\sim$20 mJy.

The \oneone/\twotwo\ ratio varies from $\sim 1$ at densities $\gtrsim10^5$ \percc\ to 14
around $10^3$ \percc.  Therefore the 2cm line may be as much as $\sim10x$ weaker than
the 6cm line.  At densities down to this limit, we should still detect the 2cm
line at $>5\sigma$ in the majority of sources and have at least marginal
detections in the weakest sources.  Above densities $\sim10^{5.5}$, the \twotwo\ line goes
into emission, so that the line ratio remains a densitometer up to ....??


We will also observe the 75-77 recombination lines of H, He, and C, which will
require no additional time but will provide information about the evolutionary
state of the sources (is a massive star present?) and will provide a measure of
the metallicity in HII regions via electron temperatures.  The 12.5 MHz
bandpasses will be placed halfway between the H and He recombination lines,
which places the lines at $\sim1/3$ and 2/3 of the way across the bandpass.  In
the pilot survey, we found that this bandwidth was enough to access continuum
on either side of the lines for baseline subtraction.

Total time including overheads will be 15 minutes for startup  + ( 10 minutes
per pointing/focus observation ) * (4 pointing/focus observations over 4 hours
using Strategy C, which recommends focus observations once per hour) + 10
minutes to observe a nearby calibrator + ( 5 minutes per scan + average 1
minute slew time + scan reset between pointings ) * 26 sources = 3.8 hours.  We
therefore request 4 hours of observing time for this proposal.  All sources are
within 2 hours of each other in RA, so only one observing session is required.

{\vspace{-3.5mm}
\bibliography{H2CO_large}
}

\end{document}

